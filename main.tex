%%%%%%%%%%%%%%%%%%%%%%%%%%%%%%%%%%%%%%%%%
% Beamer Presentation
% LaTeX Template
% Version 1.0 (10/11/12)
%
% This template has been downloaded from:
% http://www.LaTeXTemplates.com
%
% License:
% CC BY-NC-SA 3.0 (http://creativecommons.org/licenses/by-nc-sa/3.0/)
%
%%%%%%%%%%%%%%%%%%%%%%%%%%%%%%%%%%%%%%%%%

%----------------------------------------------------------------------------------------
%	PACKAGES AND THEMES
%----------------------------------------------------------------------------------------

\documentclass{beamer}

\mode<presentation> {

% The Beamer class comes with a number of default slide themes
% which change the colors and layouts of slides. Below this is a list
% of all the themes, uncomment each in turn to see what they look like.

%\usetheme{default}
%\usetheme{AnnArbor}
%\usetheme{Antibes}
%\usetheme{Bergen}
%\usetheme{Berkeley}
%\usetheme{Berlin}
%\usetheme{Boadilla}
%\usetheme{CambridgeUS}
%\usetheme{Copenhagen}
%\usetheme{Darmstadt}
%\usetheme{Dresden}
%\usetheme{Frankfurt}
%\usetheme{Goettingen}
%\usetheme{Hannover}
%\usetheme{Ilmenau}
%\usetheme{JuanLesPins}
%\usetheme{Luebeck}
%\usetheme{Madrid}
%\usetheme{Malmoe}
%\usetheme{Marburg}
%\usetheme{Montpellier}
%\usetheme{PaloAlto}
%\usetheme{Pittsburgh}
%\usetheme{Rochester}
%\usetheme{Singapore}
%\usetheme{Szeged}
\usetheme{Warsaw}

% As well as themes, the Beamer class has a number of color themes
% for any slide theme. Uncomment each of these in turn to see how it
% changes the colors of your current slide theme.

%\usecolortheme{albatross}
%\usecolortheme{beaver}
%\usecolortheme{beetle}
%\usecolortheme{crane}
%\usecolortheme{dolphin}
%\usecolortheme{dove}
%\usecolortheme{fly}
%\usecolortheme{lily}
%\usecolortheme{orchid}
%\usecolortheme{rose}
%\usecolortheme{seagull}
\usecolortheme{seahorse}
%\usecolortheme{whale}
%\usecolortheme{wolverine}

%\setbeamertemplate{footline} % To remove the footer line in all slides uncomment this line
%\setbeamertemplate{footline}[page number] % To replace the footer line in all slides with a simple slide count uncomment this line

%\setbeamertemplate{navigation symbols}{} % To remove the navigation symbols from the bottom of all slides uncomment this line
}

\usepackage{graphicx} % Allows including images
\usepackage{booktabs} % Allows the use of \toprule, \midrule and \bottomrule in tables

\usepackage{multicol}
\usepackage[portuguese]{babel}
\usepackage[utf8]{inputenc}
%----------------------------------------------------------------------------------------
%	TITLE PAGE
%----------------------------------------------------------------------------------------

\title[Aplicação de BDD em Python/Django]{Verificação e Validação de Software \\
Aplicação de BDD em Python/Django} % The short title appears at the bottom of every slide, the full title is only on the title page

\author{José Pedro, Lucas Moura, Macário Soares e Parley} % Your name
\institute[UnB] % Your institution as it will appear on the bottom of every slide, may be shorthand to save space
{
Universidade de Brasília \\ % Your institution for the title page
\medskip
%\textit{1jpsneto@gmail.com} % Your email address
}
\date{\today} % Date, can be changed to a custom date

\begin{document}

\begin{frame}
\titlepage % Print the title page as the first slide
\end{frame}

\begin{frame}
\frametitle{Agenda} % Table of contents slide, comment this block out to remove it
\tiny \begin{multicols}{2}
  \tableofcontents
\end{multicols}% Throughout your presentation, if you choose to use \section{} and \subsection{} commands, these will automatically be printed on this slide as an overview of your presentation
\end{frame}

%----------------------------------------------------------------------------------------
%	PRESENTATION SLIDES
%----------------------------------------------------------------------------------------

%------------------------------------------------
%------------------------------------------------

\section{Introdução}
\subsection{Resumo da Proposta}
\begin{frame}
  \begin{itemize}
      \item Verificação e Validação;
      \item $Behavior$ $Driven$ $Development$ (BDD);
      \item Aplicações web Python/Django;
      \item Ferramenta de aplicação de BDD em Python/Django: Lettuce;
      \item Há poucas fontes de documentação e tutoriais de uso abrangente, além
das mesmas serem fragmentadas e dispersas;
      \item Testes BDD no software Busine.me.
  \end{itemize}
\end{frame}

\subsection{Objetivos}
\begin{frame}
  \begin{itemize}
    \item Principal: desenvolver um guia de implementação BDD em português para Python/Django
usando a ferramenta Lettuce, focando contribuir na consolidação de fontes de informações
de uso.
    \item Objetivos Específicos:
        \begin{itemize}
            \item apresentar uma breve fundamentação teórica sobre Behavior Driven Development
(BDD);
            \item apresentar de forma geral o funcionamento da ferramenta Lettuce no contexto
Python/Django;
            \item desenvolver tutorial de uso da ferramenta Lettuce;
            \item apresentar a implementação de testes BDD num software em evolução Python/Django;
            \item oferecer como suporte ao guia de implementação de BDD vídeo tutorial sobre
configurações e uso da ferramenta Lettuce.
        \end{itemize}
  \end{itemize}
\end{frame}

\subsection{Dificuldades Encontradas}
\begin{frame}
  \begin{itemize}
    \item Quanto a tecnologia:
        \begin{itemize}
            \item alta curva de aprendizado sobre o framework Django;
            \item alta curva de aprendizado sobre o framework de BDD Lettuce;
            \item vários pacotes de dependências para instalação e uso do Lettuce e Django;
            \item configuração do framework Django;
            \item configuração da ferramenta Lettuce no framework Django;
            \item configuração de um browser default para o splinter.
        \end{itemize}
    \item Quanto ao processo de desenvolvimento e pesquisa:
        \begin{itemize}
            \item atrasos no cronograma;
            \item bloqueio de tarefas de cada integrante por uma depender da outra;
            \item escopo do trabalho parcialmente definido;
            \item inconsistências no relatório anterior.
        \end{itemize}
  \end{itemize}
\end{frame}

\section{Riscos Técnicos} % Sections can be created in order to organize your presentation into discrete blocks, all sections and subsections are automatically printed in the table of contents as an overview of the talk
%------------------------------------------------
\subsection{Riscos Técnicos}
\begin{frame}
  \begin{itemize}
      \item Escopo indefinido;
      \item Equipe de desenvolvimento de MDS;
      \item Inviabilidade do grupo de desenvolver aplicação em Python/Django;
      \item Inviabilidade do uso da ferramenta Lettuce para o projeto.
  \end{itemize}
\end{frame}

\section{Produto}
\subsection{Produto}
\begin{frame}
    \begin{itemize}
        \item Historias de usuário;
        \item Criterios de aceitação;
        \item Teste na ferramenta lettuce;
        \item Guia.
    \end{itemize}
\end{frame}

\section{Resultados Alcançados} % Sections can be created in order to organize your presentation into discrete blocks, all sections and subsections are automatically printed in the table of contents as an overview of the talk
%------------------------------------------------
\subsection{Resultados Alcançados}
\begin{frame}
  \begin{itemize}
      \item Testes com Letucce;
        \begin{itemize}
          \item Com e sem Splinter;
          \item Test browser.
        \end{itemize}
      \item Guia prévio de uso/instalação da Ferramenta;
      \item Motivação da equipe de MDS a utilizar BDD.
  \end{itemize}
\end{frame}

\section{Perspectivas Futuras} % Sections can be created in order to organize your presentation into discrete blocks, all sections and subsections are automatically printed in the table of contents as an overview of the talk
%------------------------------------------------
\subsection{Perspectivas Futuras}
\begin{frame}
  \begin{itemize}
      \item Finalização do guia de desenvolvimento BDD para Python/Django;
      \item Desenvolvimento do suporte de implementação com um tutorial em vídeo.
  \end{itemize}
\end{frame}



%------------------------------------------------
\section{Fim}
\subsection{Fim}
\begin{frame}
\Huge{\centerline{Obrigado!}}
\end{frame}

%----------------------------------------------------------------------------------------

\end{document} 
