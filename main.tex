%%%%%%%%%%%%%%%%%%%%%%%%%%%%%%%%%%%%%%%%%
% Beamer Presentation
% LaTeX Template
% Version 1.0 (10/11/12)
%
% This template has been downloaded from:
% http://www.LaTeXTemplates.com
%
% License:
% CC BY-NC-SA 3.0 (http://creativecommons.org/licenses/by-nc-sa/3.0/)
%
%%%%%%%%%%%%%%%%%%%%%%%%%%%%%%%%%%%%%%%%%

%----------------------------------------------------------------------------------------
%	PACKAGES AND THEMES
%----------------------------------------------------------------------------------------

\documentclass{beamer}

\mode<presentation> {

% The Beamer class comes with a number of default slide themes
% which change the colors and layouts of slides. Below this is a list
% of all the themes, uncomment each in turn to see what they look like.

%\usetheme{default}
%\usetheme{AnnArbor}
%\usetheme{Antibes}
%\usetheme{Bergen}
%\usetheme{Berkeley}
%\usetheme{Berlin}
%\usetheme{Boadilla}
%\usetheme{CambridgeUS}
%\usetheme{Copenhagen}
%\usetheme{Darmstadt}
%\usetheme{Dresden}
%\usetheme{Frankfurt}
%\usetheme{Goettingen}
%\usetheme{Hannover}
%\usetheme{Ilmenau}
%\usetheme{JuanLesPins}
%\usetheme{Luebeck}
%\usetheme{Madrid}
%\usetheme{Malmoe}
%\usetheme{Marburg}
%\usetheme{Montpellier}
%\usetheme{PaloAlto}
%\usetheme{Pittsburgh}
%\usetheme{Rochester}
%\usetheme{Singapore}
%\usetheme{Szeged}
\usetheme{Warsaw}

% As well as themes, the Beamer class has a number of color themes
% for any slide theme. Uncomment each of these in turn to see how it
% changes the colors of your current slide theme.

%\usecolortheme{albatross}
%\usecolortheme{beaver}
%\usecolortheme{beetle}
%\usecolortheme{crane}
%\usecolortheme{dolphin}
%\usecolortheme{dove}
%\usecolortheme{fly}
%\usecolortheme{lily}
%\usecolortheme{orchid}
%\usecolortheme{rose}
%\usecolortheme{seagull}
\usecolortheme{seahorse}
%\usecolortheme{whale}
%\usecolortheme{wolverine}

%\setbeamertemplate{footline} % To remove the footer line in all slides uncomment this line
%\setbeamertemplate{footline}[page number] % To replace the footer line in all slides with a simple slide count uncomment this line

%\setbeamertemplate{navigation symbols}{} % To remove the navigation symbols from the bottom of all slides uncomment this line
}

\usepackage{graphicx} % Allows including images
\usepackage{booktabs} % Allows the use of \toprule, \midrule and \bottomrule in tables

\usepackage{multicol}
\usepackage[portuguese]{babel}
\usepackage[utf8]{inputenc}
%----------------------------------------------------------------------------------------
%	TITLE PAGE
%----------------------------------------------------------------------------------------

\title[Estudo de caso de BDD]{Aplicação de BDD em aplicações web Python/Django} % The short title appears at the bottom of every slide, the full title is only on the title page

\author{José Pedro, Lucas Moura, Macário Soares e Parley} % Your name
\institute[UnB] % Your institution as it will appear on the bottom of every slide, may be shorthand to save space
{
Universidade de Brasília \\ % Your institution for the title page
\medskip
%\textit{1jpsneto@gmail.com} % Your email address
}
\date{\today} % Date, can be changed to a custom date

\begin{document}

\begin{frame}
\titlepage % Print the title page as the first slide
\end{frame}

\begin{frame}
\frametitle{Agenda} % Table of contents slide, comment this block out to remove it
\tiny \begin{multicols}{2}
  \tableofcontents
\end{multicols}% Throughout your presentation, if you choose to use \section{} and \subsection{} commands, these will automatically be printed on this slide as an overview of your presentation
\end{frame}

%----------------------------------------------------------------------------------------
%	PRESENTATION SLIDES
%----------------------------------------------------------------------------------------

%------------------------------------------------
%------------------------------------------------

\section{Contexto}
\subsection{Contexto}
\begin{frame}
  \begin{itemize}
      \item Verificação e Validação;    
      \item $Behavior$ $Driven$ $Development$ (BDD);
      \item Aplicações web Python/Django.
  \end{itemize}
\end{frame}

\section{Objetivos}
\subsection{Objetivos}
\begin{frame}
  \begin{itemize}
    \item Principal: desenvolver um estudo de caso usando Behavior Driven Development (BDD) no contexto de aplicação web em python/Django;
    \item Objetivos Específicos:
        \begin{itemize}
            \item selecionar e analisar ferramentas disponíveis de BDD para Python/Django;
            \item implementar testes de BDD num software em desenvolvimento;
            \item desenvolver um guia de boas práticas de BDD no contexto Python/Django.
        \end{itemize}
  \end{itemize}
\end{frame}

\section{Fundamentação Teórica} % Sections can be created in order to organize your presentation into discrete blocks, all sections and subsections are automatically printed in the table of contents as an overview of the talk
%------------------------------------------------
\subsection{Fundamentação Teórica}
\begin{frame}
  \begin{itemize}
      \item História;  
      \item Conceitos;
      \item Definição.
  \end{itemize}
\end{frame}

\section{Objetos de Estudo}
\subsection{Objetos de Estudo}
\begin{frame}
    \begin{itemize}
        \item Técnica;
        \item Ferramenta;
        \item Boas práticas.
    \end{itemize}
\end{frame}

\section{Caso Selecionado} % Sections can be created in order to organize your presentation into discrete blocks, all sections and subsections are automatically printed in the table of contents as an overview of the talk
%------------------------------------------------
\subsection{Caso Selecionado}
\begin{frame}
  \begin{itemize}
      \item Busine.me;  
      \item GPP e MDS;
      \item Lei de Acesso a Informação Pública.
  \end{itemize}
\end{frame}

\section{Restrições} % Sections can be created in order to organize your presentation into discrete blocks, all sections and subsections are automatically printed in the table of contents as an overview of the talk
%------------------------------------------------
\subsection{Restrições}
\begin{frame}
  \begin{itemize}
      \item Estudo realizado para plataforma específica;  
      \item Consistência dos dados;
      \item Pouca população amostral.
  \end{itemize}
\end{frame}

\section{Critérios} % Sections can be created in order to organize your presentation into discrete blocks, all sections and subsections are automatically printed in the table of contents as an overview of the talk
%------------------------------------------------
\subsection{Critérios}
\begin{frame}
  \begin{itemize}
      \item Quantidade de erros;  
      \item Padrão dos erros;
      \item Eficiência da ferramenta.
  \end{itemize}
\end{frame}






%------------------------------------------------
\section{Fim}
\subsection{Fim}
\begin{frame}
\Huge{\centerline{Obrigado!}}
\end{frame}

%----------------------------------------------------------------------------------------

\end{document} 
