\section{Gerenciamento do Projeto}
\subsection{Planejamento}
Para o desenvolvimento do projeto, de maneira a integrar todas as engenharias envolvidas e não individualizar o trabalho de cada integrante dentro de sua especialidade, foi feita uma divisão em três frentes de trabalho: uma responsável pela estrutura física do protótipo, outra para trabalhar com os algoritmos e programação envolvidos no sistema, e por último, uma voltada para desenvolver o sistema de limpeza. De tal forma que haja integrantes de mais de uma engenharia em cada frente, mas que ao mesmo tempo não haja exclusão de ocupação, ou seja, um mesmo integrante pode trabalhar em mais de uma frente, para que o desenvolvimento do trabalho seja feito sempre em equipe.
Assim, a divisão geral para a evolução do trabalho nesta primeira etapa se deu da seguinte forma:

\begin{table}[h!]
  \begin{center}
    \begin{tabular}{|l|l|l|}
      \hline
      Estrutura & Controle e Eletrônica & Método de Limpeza \\
      \hline
      João (automotiva) & Lucas (software) & Marcos (software) \\
      Rodrigo (software) & Matheus (eletrônica) & Vanessa (energia) \\
      Thaynara (energia) & Pablo (eletrônica) & José Armando (eletrônica) \\
      Tuane (energia) & Maria (energia) & José Pedro (software) \\
      \hline
    \end{tabular}
  \end{center}
\end{table}


\subsection{Cronograma}
Tendo como objetivo final a apresentação de um protótipo funcional ao final da disciplina e considerando três apresentações ao longo desse percurso, o cronograma do processo de produção, já executado até agora e o planejado, encontra-se a seguir:

Quarta
Sexta
Quarta
Sexta
Quarta
Sexta
Quarta
Sexta


\begin{table}[h!]
  \begin{center}
    \begin{tabular}{|l|l|l|l|}  
       Semana 1 & Semana 2 & Semana 3 & Semana 4
      \begin{tabular}{|l|l|l|l|l|l|l|l|}
        \hline
        Quarta & Sexta & Quarta & Sexta & Quarta & Sexta & Quarta & Sexta 
        
        
        
      \end{tabular}
    \end{tabular}  
  \end{center}
\end{table}

% FAZER TABELA


\subsection{Custos}
******* TO DO ********